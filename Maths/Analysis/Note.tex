\documentclass[12pt, a4paper, UTF8, openany]{ctexbook}

\usepackage{geometry}%设置页边距
\geometry{left=1.54cm, right=1.54cm, top=2.18cm, bottom=2.18cm}

\usepackage{tikz}%绘图

\usepackage{amsmath, amssymb}%数学

\usepackage{array}%表格基础设置
\usepackage{longtable}%跨页表格
\usepackage{booktabs}%表格水平线

\usepackage[colorlinks = true, 
            linkcolor = black]{hyperref}%超链接跳转

\newcommand{\dif}{\mathrm{d}}
\newcommand{\difx}{\,\mathrm{d}x}

\newcommand{\difk}[2]{{#1}^{(#2)}}

\newcommand{\field}[1]{\mathbb{#1}}

\newcommand{\imunit}{\mathrm{i}}%虚数单位

\linespread{1.05}%行间距
\setlength{\parskip}{8pt}%段落间距
\setlength{\parindent}{0pt}%不缩进

\title{数学分析}
\author{1001}

\begin{document}
    \maketitle

    %未知错误
    % \newpage
    \tableofcontents

    \chapter{函数,极限,连续}

    \chapter{一元函数微分}

    \section{微分中值定理}

    \section{L'Hôpital 法则}

    \section{Taylor 定理}
    多项式拟合函数,记 $$P_n(x) = \sum_{k = 0}^{n} \frac{\difk{f}{k}(x_0)}{k!}(x - x_0)^k$$
    $R_n(x) = f(x) - P_n(x)$。

    对 $P_n(x)$ 求 $k$ 阶导,则所有幂次 $t < k$ 的项为零,
    所有幂次 $t \geqslant k$ 的项形如 $\dfrac{\difk{f}{t}(x_0)}{(t - k)!}(x - x_0)^{t - k}$。
    幂次永远与阶乘相同。

    总有 $\forall k < n, 
    \difk{R_n}{k}(x_0) = \difk{f}{k}(x_0) - \difk{f}{k}(x_0) 
    - \sum_{t > k}\frac{\difk{f}{t}(x_0)}{(t - k)!}(x_0 - x_0)^{t - k} = 0$
    
        \subsection{误差估计}
        Peano 余项:$$R_n(x) = o((x - x_0)^n)$$
        证明:$f(x)$ 在 $x_0$ 处有 $n$ 阶导数,则只保证 $x_0$ 领域内有 $1\sim(n - 1)$ 阶连续导数,只能用 $n - 1$ 次洛必达法则。 \\
        $\begin{aligned}
            \lim_{x \to x_0} \frac{R_n(x)}{(x - x_0)^n} &= \mathop{\cdots}\limits^{\text{L'Hôpital}} = 
            \lim_{x \to x_0} \frac{\difk{R_n}{n - 1}(x)}{n!(x - x_0)} \\
            &= \frac{1}{n!}\lim_{x \to x_0} \frac{\difk{f}{n - 1}(x) - \difk{f}{n - 1}(x_0) - \difk{f}{n}(x_0)(x - x_0)}{x - x_0} \\
            &= \frac{1}{n!} \lim_{x \to x_0} 
                \left[\frac{\difk{f}{n - 1}(x) - \difk{f}{n - 1}{x_0}}{x - x_0} - \difk{f}{n}{x_0}\right]
        \end{aligned}$\\
        最后一个等号源自 $f^{n}(x_0)$ 的定义。

        Lagrange 余项:
        $$R_n(x) = \frac{\difk{f}{n + 1}(\xi)}{(n + 1)!}(x - x_0)^{n + 1}, \; \xi \in x \sim x_0$$
        证明:
        $R_n(x) = \dfrac{\difk{f}{n + 1}(\xi)}{(n + 1)!}(x - x_0)^{n + 1} 
        \iff \dfrac{R_n(x)}{(x - x_0)^{n + 1}} = \dfrac{\difk{f}{n + 1}(\xi)}{(n + 1)!}$ \\
        连用 $n + 1$ 次 Cauchy 中值定理,每次从区间内取 $\xi_k$。\\
        $\begin{aligned}
            \dfrac{R_n(x)}{(x - x_0)^{n + 1}} &= \dfrac{R_n(x) - R_n(x_0)}{(x - x_0)^{n + 1} - (x_0 - x_0)^{n + 1}} = 
            \dfrac{\difk{R_n}{1}(\xi_1)}{\difk{((x - x_0)^n)}{1}}
            = \mathop{\cdots}\limits^{\text{如法炮制}} \\
            &= \dfrac{\difk{R_n}{n}(\xi_n) - \difk{R_n}{n}(x_0)}{(n + 1)!(x - x_0) - (x_0 - x_0)} = 
            \dfrac{\difk{f}{n + 1}(\xi)}{(n + 1)!}
        \end{aligned}$

        改写 Lagrange 余项(注意 $\theta$ 取值为开区间,可取端点):
        $$R_n(x) = \dfrac{\difk{f}{n + 1}\left[x_0 + \theta(x - x_0)\right]}{(n + 1)!}(x - x_0)^{n + 1}, \; \theta \in (0, 1)$$

        积分余项(注意阶乘为 $n$ 而非 $n + 1$):
        $$R_n(x) = \dfrac{\int_{0}^{1} (1 - t)^n\difk{f}{n + 1}(x_0 + t(x - x_0))\,\dif t}{n!} (x - x_0)^{n + 1}$$

        Cauchy 余项(注意 $\theta$ 取值为闭区间,可能仅能取端点):
        $$R_n(x) = \dfrac{(1 - \theta)^n\difk{f}{n + 1}(x_0 + \theta(x - x_0))}{n!}(x - x_0)^{n + 1}, \; \theta \in [0, 1]$$
        证明:对积分余项用积分中值定理。

        \subsection{Maclaurin 公式}
        $x_0 = 0, \, R_n(x) = \dfrac{\difk{f}{n + 1}(\theta x)}{(n + 1)!}x^{n + 1}$

        \begin{longtable}[h]{w{l}{12cm}}
            $\exp x = 1 + x + \dfrac{x^2}{2} + \dfrac{x^3}{3!} + \cdots + \dfrac{x^n}{n!} + \cdots$ \\
            \addlinespace
            $\ln(1 + x) = x - \dfrac{x^2}{2} + \dfrac{x^3}{3} - \dfrac{x^4}{4} 
                + \cdots + (-1)^{k + 1}\dfrac{x^k}{k} + \cdots$ \\
            \midrule
            $\sin x = x - \dfrac{x^3}{3!} + \dfrac{x^5}{5!} - \dfrac{x^7}{7!} 
                        + \cdots + (-1)^{k}\dfrac{x^{2k + 1}}{(2k + 1)!} + \cdots$ \\
            \addlinespace
            $\cos x = 1 - \dfrac{x^2}{2!} + \dfrac{x^4}{4!} - \dfrac{x^6}{6!}
                        + \cdots + (-1)^{k}\dfrac{x^{2k}}{(2k)!} + \cdots$ \\
            \midrule
            $(1 + x)^\alpha = 1 + \alpha x + \dbinom{\alpha}{2}x^2 + \dbinom{\alpha}{3}x^3 
                + \cdots + \dbinom{\alpha}{k}x^k + \cdots$ \\
            \midrule
            $\begin{aligned}
                \dfrac{1}{1 + x} &= \dfrac{\dif}{\dif x}\ln(1 + x) \\
                    &= 1 - x + x^2 - x^3 + \cdots + (-1)^kx^k + \cdots
            \end{aligned}$ \\
            $\dfrac{1}{1 - x} = 1 + x + x^2 + x^3 + \cdots + x^k + \cdots$ \\
            \addlinespace
            $\sqrt{1 + x} = 1 + \dfrac{1}{2}x - \dfrac{1}{8}x^2 + \cdots 
                + (-1)^{k + 1}\dfrac{(2k - 3)!!}{(2k)!!}x^k + \cdots$ \\
            \addlinespace
            $\dfrac{1}{\sqrt{1 + x}} = 1 - \dfrac{1}{2}x + \dfrac{3}{8}x^2 
                - \cdots + (-1)^k\dfrac{(2n - 1)!!}{(2n)!!}x^n + \cdots$
        \end{longtable}
        
        \subsection{应用}
        定量估计误差:Lagrange 余项。

        近似计算。

        求极限 / 高阶导数值:展开成多项式。

        函数比大小:Lagrange 余项。适度展开,计算点值,etc。

    \section{函数性质研究}
        \subsection{单调性}

        \subsection{极值与最值}
        驻点:$f'(x_0) = 0$ 的点 $x = x_0$。

        判定:
        \begin{itemize}
            \item $f'(x)$ 变号:\\
                a. 由负到正:极小值;\\
                b. 由正到负:极大值。
            \item $f'(x_0) = 0$,利用 $f''(x_0)$:\\
                a. $f''(x_0) > 0$:极小值;\\
                b. $f''(x_0) < 0$:极大值。
            \item 推广:$\difk{f}{1}(x_0) = \cdots = \difk{f}{k - 1}(x_0) = 0, \, \difk{f}{k}(x_0) \neq 0$:\\
                {\heiti 当且仅当 $k$ 为偶数}时,为极值;判定同二阶导。
        \end{itemize}
        证明:泰勒展开,Peano 余项即可。

        极值点还可能包括{\heiti 不可导点},但{\heiti 边界点依定义不是}。

        最值点:极值点和边界点。

        \subsection{凹凸性}
        {\heiti 上凸 = 凸} = 下凹:从上/下方观察,曲线向 $y$ 轴正半轴方向凸出。

        {\heiti 上凹 = 凹} = 下凸:从上/下方观察,曲线向 $y$ 轴负半轴方向凸出。

        上凹的等价定义(上凸函数取反即得上凹函数):
        \begin{itemize}
            \item $\forall P_1, P_2$,线段 $\overline{P_1P_2}$ 在弧 $\overset{\frown}{P_1P_2}$ 上方。
            \item $\forall x_0, (x_0, f(x_0))$ 处切线在曲线下方。
            \item $\forall x_1, x_2 \; f(\frac{x_1 + x_2}{2}) < \frac{1}{2}(f(x_1) + f(x_2))$。
            \item $\forall x_1, x_2, \lambda \in (0, 1) \; 
                f(\lambda x_1 + (1 - \lambda)x_2) < \lambda f(x_1) + (1 - \lambda)f(x_2)$。
        \end{itemize}

        可证明凸函数除了端点以外,为连续函数。

        凹凸性判定:$\forall x \in I$:\\
        a. $\difk{f}{2}(x) > 0$:上凹。 \\
        b. $\difk{f}{2}(x) < 0$:上凸。 \\
        证明:依据中点凸定义,在中点处泰勒展开,Peano 余项即可。
        
        拐点:凹凸性改变的点 $(x, y)$,$x$ 是{\heiti 一阶导极值点}。\\
        拐点判定:$\difk{f}{2}(x) = 0$,必要{\heiti 不充分}。见极值点判定。

        \subsection{渐近线}
        水平 / 铅直:趋于 $\pm \infty$ / 奇点。

        斜:$$\begin{aligned}
            k &= \lim_{x \to \infty} \frac{f(x)}{x} \\
            b &= \lim_{x \to \infty} (f(x) - kx)
        \end{aligned}$$

        \subsection{曲率}
        $$k := \lim_{\Delta s \to 0} \left| \frac{\Delta\alpha}{\Delta s} \right| = \left|\frac{\dif\alpha}{\dif s}\right|$$

        对于 $y = y(x)$,则 $\alpha = \alpha(x), s = s(x)$。具体地:

        $\alpha$:
        $\tan\alpha = y' \implies \alpha = \arctan y' \implies \dfrac{\dif\alpha}{\dif x} = \dfrac{y''}{1 + (y')^2}$。

        $s$:
        $\dif s = \sqrt{(\dif x)^2 + (\dif y)^2} \implies \dfrac{\dif s}{\dif x} = \sqrt{1 + (y')^2}$。

        $$k = \left|\frac{\dif\alpha}{\dif s}\right| = \left| \frac{y''}{\left( 1 + (y')^2 \right)^{\frac{3}{2}}} \right|$$

        曲率半径 $\rho := \frac{1}{k}$。\\
        曲率圆:圆心在曲线凹向,与曲线相切(即共切线)。

        \subsection{作图}

    \chapter{一元函数积分}
    \section{常义}
        \subsection{定义}
        $$\int_{a}^{b} f(x)\dif x := \lim\limits_{d = \max\{\Delta x\} \to 0} \sum f(\xi_i)\Delta x_i$$

        几何意义:面积代数和。定义:$\int_{b}^{a} := -\int_{a}^{b},\; \int_{a}^{a} := 0$

        记 $M_k = \sup\{f(x): x \in [x_{k - 1}, x_k]\}, \; m_k = \inf\{f(x): x \in [x_{k - 1}, x_k]\}$。
        定义 Darboux 大和 $S_n = \sum M_k\Delta x_k$,Darboux 小和 $s_n = \sum m_k\Delta x_k$。
        则 Riemann 可积的充要条件是 $\lim_{d \to 0} S_n - s_n = 0$。

        注:积分变量任意。$\int_{a}^{x} f(t)\dif t = \int_{a}^{x} f(x)\dif x$。

        \subsection{可积}
        \begin{itemize}
            \item $f\in C[a, b]$。\\
            证明:连续则{\heiti 一致连续},则存在 $d$ 使 $\forall |u - v| < d, |f(u) - f(v)| < \epsilon$。
            \item {\heiti 有界}函数 $f$ 在 $[a, b]$ 有有限个第一类间断点。
            \item {\heiti 有界}函数 $f$ 在 $[a, b]$ 单调。
        \end{itemize}
        
        \subsection{性质}
        \begin{itemize}
            \item 线性:$\int_{a}^{b} (\lambda f(x) + \mu g(x))\dif x = \lambda\int_{a}^{b}f(x)\dif x + \mu\int_{a}^{b}g(x)\dif x$。
            \item 区间可加:$\int_{a}^{b} = \int_{a}^{c} + \int_{c}^{b}$。
            \item 保序:$f(x) \leqslant g(x) \; \implies \; \int_{a}^{b} f(x)\dif x \leqslant \int_{a}^{b} g(x)\dif x$。\\
                推论:$m(b - a) \leqslant \int_{a}^{b} f(x)\dif x \leqslant M(b - a)$;
                $|\int_{a}^{b} f(x)\dif x| \leqslant \int_{a}^{b} |f(x)|\dif x$。\\
                {\heiti 应用}:估值。
            \item 积分中值定理:$f(x)$ {\heiti 连续},$g(x)$ 不变号,
                则 $\exists \xi \in [a, b], \; \int_{a}^{b} f(x)g(x)\dif x = f(\xi)\int_{a}^{b} g(x)\dif x$。\\
                证明:不妨 $g \geqslant 0$,$m\int_{a}^{b} g \leqslant \int_{a}^{b} fg \leqslant M\int_{a}^{b} g 
                \implies m \leqslant (\int_{a}^{b} fg) / (\int_{a}^{b} g) \leqslant M$。由拉格朗日中值即得。\\
                推论:$\exists \xi, \; \int_{a}^{b} f(x)\dif x = f(\xi)(b - a)$。
        \end{itemize}

        \subsection{原理}
        \begin{itemize}
            \item Newton - Leibniz 公式:$$\int_{a}^{b} f(x)\dif x = F(x)|_a^b$$\\
                证明:$F(b) - F(a) = \sum F(x_k) - F(x_{k - 1}) = \sum F'(\xi_k)\Delta x_k 
                    = \sum f(\xi_k)\Delta x_k = \int_{a}^{b} f(x)\dif x$。
        \end{itemize}
        
        积分上限函数:$\Phi(x) = \int_{a}^{x} f(t)\dif t$。基本定理:
        \begin{itemize}
            \item 第一:$f \in C[a, b]$,则 $$\Phi(x)' = \frac{\dif}{\dif x}\int_{a}^{x} f(t)\dif t = f(x)$$
                {\heiti 说明连续函数必有原函数}。\\
                证明:$\Delta\Phi(x) = \int_{x}^{x + \Delta x} f(t)\dif t = f(\xi)\Delta x$。
            \item 第二:$F + C$ 为所有原函数。
        \end{itemize}

        {\heiti 积分上限求导}:
        $$\Psi(x) = \int_{u(x)}^{v(x)} f(t)\dif t = -\int_{p}^{u(x)} + \int_{p}^{v(x)} 
        \implies \frac{\dif}{\dif x}\Psi(x) = -f(u(x))u'(x) + f(v(x))v'(x)$$ 
        为复合函数 $\Phi(u/v)$。若 $f(t)$ 为 $f(x, t)$,应先拆出 $x$。

        
        
    \section{计算}
        \subsection{积分法}
        \begin{itemize}
            \item 换元:$$\dif F(\phi(x)) \iff F'(\phi(x))\,\dif\phi(x)$$
                左到右:换元 $x = u(x)$,并且 $u(x)$ {\heiti 单调};右到左:凑微分。\\
                定积分换元:换变量同时换上下限。
            \item 分布:$$\dif uv = v\dif u + u\dif v \; \implies \; \int v\dif u = uv - \int u\dif v$$
                定积分分布:相同的上下限 $|_a^b,\, \int_{a}^{b}$。
        \end{itemize}

        \subsection{幂函数}
        $$\int x^\alpha\,\dif x = 
            \begin{cases}
                \frac{1}{\alpha + 1}\, x^{\alpha + 1} + C &\alpha\neq -1 \\
                \ln|x| + C &\alpha = -1
            \end{cases}$$

        \subsection{指数与对数}
        $$\int \alpha^x\,\dif x = \frac{\alpha^x}{\ln\alpha} + C \quad \alpha > 0 \land \alpha \neq 1$$\\
        特别地,$$\int e^x\,\dif x = e^x + C$$

        $$\int\ln x\difx = x\ln x - x + C$$

        $e^x$ 与常数的分式:强行提出 $e^x$ 使 $\difx \to \dif t := \dif e^x$。换元化为有理函数。

        $\int e^x(f(x) + f'(x))\difx = e^xf(x) + C$。

        \subsection{(反)三角函数}
        \begin{itemize}
            \item 基础
                \begin{longtable}[h]{*{2}{w{l}{8cm}}}
                    $\int \sin x\,\dif x = -\cos x + C$ & $\int \cos x\,\dif x = \sin x + C$\\
                    &\\
                    $\int \cot x\,\dif x = -\ln|\sin x| + C$ & $\int \tan x\,\dif x = -\ln|\cos x| + C$\\
                    $\int \csc^2 x\difx = -\cot x + C$ & $\int \sec^2 x\difx = \tan x + C$\\
                    &\\
                    $\int \csc x\difx = -\ln|\csc x + \cot x| + C$ & $\int \sec x\difx = \ln|\sec x + \tan x| + C$\\
                    $\int \csc x \cot x\difx = -\csc x + C$ & $\int \sec x \tan x\difx = \sec x + C$\\
                    &\\
                    $\int \frac{\dif x}{\sqrt{1 - x^2}} = \arcsin x + C = -\arccos x + C$ & 
                    $\int \frac{\dif x}{1 + x^2} = \arctan x + C$
                \end{longtable}
                证明:$\int \tan$:换元 $\sin x\difx = -\dif\cos x$。\\
                    $\int \sec$:$(\sec + \tan)' = \sec\tan + \sec^2 = \sec(\sec + \tan) 
                                \implies \sec = \frac{(\sec + \tan)'}{\sec + \tan}$\\
                    或:$\int\sec\difx = \int \frac{\cos x}{\cos^2 x}\dif x = \int \frac{\dif\sin x}{1 - \sin^2 x}$
            
            \item 万能代换
                $$\begin{aligned}
                    t &= \tan\frac{x}{2}\quad &\dif x &= \frac{2}{1 + t^2}\dif t\\
                    \sin x &= \frac{2t}{1 + t^2}\quad &\cos x &= \frac{1 - t^2}{1 + t^2}
                \end{aligned}$$\\
                证明:$x = 2\arccos t$;二倍角公式展开,除以 $1 = \sin^2 x + \cos^2 x$,同除 $\cos^2 x$。

            \item 分式\\
                分母齐次:提 $\cos^2 x$,凑 $\dif x \to \dif\tan x$。利用二倍角凑出二次式以便换元。\\
                $\int \dfrac{a\cos x + b\sin x}{c\cos x + d\sin x}\dif x$:
                    拆分子为 $\lambda(c\cos x + d\sin x) + \mu(c\cos x + d\sin x)'$。
        \end{itemize}

        \subsection{换元}
        $\dfrac{1}{a^2 + x^2}, \dfrac{1}{\sqrt{a^2 - x^2}} \; \implies \; \arctan, \arcsin$。

        $\sqrt[n]{ax + b}$:令 $t = \sqrt[n]{ax + b}$,反解 $x$。

        $\sqrt{x^2 \pm a^2}, \sqrt{a^2 \pm x^2}$:构造直角三角形,其中一边为 $a$。换元 $\theta$,以三角函数表示其他边。\\
        二次式:配方化为上述。

        $(x \pm \frac{1}{x})' = 1 \pm \frac{1}{x^2}$,除以 $x^2$ 构造 
        $(x \pm \frac{1}{x}),\, (x \pm \frac{1}{x})^2, \, (1 \pm \frac{1}{x^2}), \, \mathrm{etc.}$

        \subsection{分布}
        优先级:反函数 $\to$ 对数 $\to$ 幂函数 $\to$ 三角函数 $\to$ 指数函数\\
        在 $u\dif v$ 中,前面的优先视为 $u$ 来求导。

        间接积分:
        \begin{itemize}
            \item $\sin, \cos$:多次求导会恢复,继而可列方程求解(如果积分存在)。\\
                此外,等式两侧同时消去 $\int f(x)\difx$ 会留下一个常数 $C$,本质是函数族。
            \item 递归:构造同构
                $$\begin{aligned}
                    I_n &= \int \tan^n\difx = \int \tan^{n - 2}x (\sec^2 - 1)\difx
                        = \int \tan^{n - 2}x\,\dif\tan x - \int\tan^{n - 2}x\difx \\
                    \implies &I_n = \frac{1}{n - 1}\tan^{n - 1}x - I_{n - 2}
                \end{aligned}$$
                
                $$\begin{aligned}
                    J_n &= \int_{0}^{\pi / 2} \sin^n\difx = \int_{0}^{\pi / 2} \cos^n\difx \text{ 证明:换元。} \\
                    J_n &= \int_{0}^{\pi / 2} \sin^{n - 2}x (1 - \cos^2 x)\difx 
                        = J_{n - 2} - \int_{0}^{\pi / 2} \sin^{n - 2}x \cos x \,\dif\sin x \\
                        &= J_{n - 2} - \int_{0}^{\pi / 2} \cos x \,\dif \frac{1}{n - 1}\sin^{n - 1} x \\
                        &= J_{n - 2} - \frac{1}{n - 1}\cos x \sin^{n - 1} x |_{0}^{\pi / 2} 
                            - \frac{1}{n - 1}\int_{0}^{\pi / 2} \sin^n x\difx \\
                    \implies &J_n = \frac{n - 1}{n} J_{n - 2}\\
                    \implies &J_n = \frac{(n - 1)!!}{n!!} J_{n \bmod 2} \quad (J_0 = \pi / 2, \; J_1 = 1)
                \end{aligned}$$
        \end{itemize}

        \subsection{有理函数}
        有理真分式 $R(x) = P(x) / Q(x)$,不妨 $Q(x)$ 为首一多项式,\\
        在 $\field{F} = \field{R}$ 分解分母得 $Q(x) = \prod (x - a)^{k} \, \prod (x^2 + px + q)^{r}$。
        其中二次式对应一对共轭复根。\\
        则 $R(x)$ 总能分解为 $$R(x) = \sum\sum_{t = 1}^k \frac{A}{(x - a)^t} + \sum\sum_{t = 1}^r \frac{Ax + B}{(x^2 + px + q)^t}$$
        分式个数至多与 $Q(x)$ 的因式个数相同。

        为求分解,可待定系数,或凑。多项式除法。

        对每一类分别积分即可:
        \begin{itemize}
            \item $\int \frac{\dif x}{(x - a)^t}$:同幂函数。
            \item $\int \frac{Ax + B}{(x^2 + px + q)^t}$:\\
                先拆分:$Ax + B = \lambda(2x + p) + \mu = \lambda(x^2 + px + q)' + \mu$,
                对二次式整体换元,为幂函数。\\
                复根则有 $\Delta < 0$,故配方得 $x^2 + px + q = (x + \phi)^2 + v^2$。
                换元 $u = x + \phi$。\\
                后者化为 $I_t = \int\frac{\dif u}{(u^2 + v^2)^t}$。

                $$\begin{aligned}
                    I_n &= \frac{u}{(u^2 + v^2)^n} - \int u \frac{-n \cdot 2u}{(u^2 + v^2)^{n + 1}}\dif u
                        = \frac{u}{(u^2 + v^2)^n} + 2n\int \frac{u^2}{(u^2 + v^2)^{n + 1}}\dif u\\
                        &= \frac{u}{(u^2 + v^2)^n} + 2n\int \frac{(u^2 + v^2) - v^2}{(u^2 + v^2)^{n + 1}}\dif u\\
                        &= \frac{u}{(u^2 + v^2)^n} + 2nI_n - 2nv^2I_{n + 1}\\
                    \implies &I_{n + 1} = \frac{u}{2nv^2(u^2 + v^2)^n} + \frac{2n - 1}{2nv^2}I_n 
                    \quad I_1 = \frac{1}{v}\arctan\frac{u}{v} + C
                \end{aligned}$$
        \end{itemize}

        \subsection{特殊函数}
        对称区间上奇偶函数定积分:显然。注意需要为{\heiti 常义积分}。

        周期函数:一个周期内定积分不变。证明:换元使整体平移。

        复杂三角函数:先限制到一个周期,证明一个周期的积分为 0。\\
        方向:$f(x) + f(-x) = 0, \; f(x) + f(x + \pi) = 0$

    \section{广义积分}
        \subsection{定义}
        无穷积分:$$\int_{a}^{+\infty} f(x)\difx := \lim_{b \to +\infty} \int_{a}^{b} f(x)\difx$$
        若存在{\heiti 有穷}极限则称原式收敛,否则发散。\\
        {\heiti 无上下界积分}:
        $\exists t, \; \int_{-\infty}^{+\infty} = \int_{t}^{+\infty} + \int_{-\infty}^{t}$ {\heiti 均收敛}则称收敛。
        易知收敛时,$t$ 的选取没有任何影响。

        瑕积分:在 $x = a$ 附近无界 
        $$\int_{a}^{b} f(x)\difx := \lim_{\varepsilon \to 0^+} \int_{a + \varepsilon}^{b} f(x)\difx$$
        若存在{\heiti 有穷}极限则称原式收敛,否则发散。\\
        {\heiti 区间内的瑕点积分}:
        $\int_a^b = \lim_{\varepsilon \to 0^-} \int_{a}^{c + \varepsilon} + \lim_{\delta \to 0^+} \int_{c + \delta}^b$ 
        均收敛则称收敛。{\heiti 两个极限过程独立}。

        瑕积分可转化为无穷积分,换元 $a + \frac{1}{t}$。

        \subsection{性质}
        对定积分成立的性质均成立。

        Newton - Leibniz 公式:问题归为原函数在无穷远处 / 瑕点处是否收敛 / 有单侧极限。

        \subsection{审敛法}
        准则对任意一种广义积分均成立。

        比较准则 I:$0 \leqslant f(x) \leqslant g(x)$,有:\\
        a. $\int g(x)\difx$ 收敛,则 $\int f(x)\difx$ 收敛;\\
        b. $\int f(x)\difx$ 发散,则 $\int g(x)\difx$ 发散。

        比较准则 II:$f(x) \geqslant 0, \, g(x) > 0$,记趋于远处/瑕点时 $\lim \frac{f(x)}{g(x)} = \lambda$,有:\\
        a. $\lambda > 0$:$\int f(x)\difx$ 和 $\int g(x)\difx$ 同时收敛或同时发散;\\
        b. $\lambda = 0$:(说明 $f << g$,)$\int g(x)\difx$ 收敛则 $\int f(x)\difx$ 收敛;\\
        c. $\lambda = +\infty$:(说明 $f >> g$),$\int g(x)\difx$ 发散则 $\int f(x)\difx$ 发散。\\
        (总之,用分母的 $g$ 的收敛性判定分子的 $f$。二者的积分上限函数均单增,故有上界则有极限。)

        绝对收敛准则:若 $\int |f(x)|\difx$ 收敛,则 $\int f(x)\difx$ 收敛,称绝对收敛。\\
        若 $\int |f(x)|\difx$ 发散,但 $\int f(x)\difx$ 收敛,称条件收敛。\\
        证明:$0 \leqslant f(x) + |f(x)| \leqslant 2|f(x)|$,由极限的四则运算知收敛。

        实践上,取 $g(x) = 1 / x^p$。$p = 1$ 恒发散,不同积分的敛散性相反。
        \begin{longtable}[h]{w{c}{2cm}*{2}{|w{c}{6cm}}}
            $\int_I \frac{\dif x}{x^p}$ & 瑕积分 $I = (0, a]$ & 无穷积分 $I = [a, \infty)$ \\
            \midrule
            $0 < p \leqslant 1$         & 发散                & 收敛 \\
            \midrule
            $1 < p$                     & 收敛                & 发散 
        \end{longtable}

        比较 $f(x)$ 与 $\frac{C}{x^p} / \frac{C}{(x - a)^p}$ 或讨论 $\lim x^pf(x) / (x - a)^pf(x)$。
        
    \section{应用}
    \begin{itemize}
        \item 级数求极限:除 $n$ 使 $\sum \Delta = C$。
        \item 弧长:$\dif l = \sqrt{(\dif x)^2 + (\dif y)^2}$\\
            极坐标:换元 $x = \rho\cos\theta, y = \rho\sin\theta$,求导按乘积求导,\\
                有 $\dif l = \sqrt{[\rho'(\theta)]^2 + [\rho(\theta)]^2}\,\dif\theta$。
        \item 面积:\\
            极坐标:扇形近似 $\rho = \rho(\theta), \, \dif S = \frac{1}{2} \rho^2 \,\dif\theta$。
        \item 体积:\\
            层切 $\dif V = \pi r^2 \,\dif h$;柱剥:$\dif V = 2\pi h \,\dif r$。
    \end{itemize}

    \chapter{微分方程}
    \section{定义}

    \section{解的结构}

    \section{一阶}
        \subsection{分离变量}

        \subsection{齐次}

        \subsection{一阶线性}
        形式:$$\frac{\dif y}{\dif x} + P(x)y = Q(x)$$

        先齐次:$\dfrac{\dif y}{\dif x} + P(x)y = 0 \implies \dfrac{\dif y}{y} = -P(x)\difx 
        \implies y = C\exp\left[-\int P(x)\difx\right]$

        再非齐次:记 $\lambda = \exp\left[-\int P(x)\difx\right]$,$y = u(x)\lambda(x)$ 带入\\
        $\begin{aligned}
            u'\exp - P\lambda u + Pu\lambda = Q &\implies \dfrac{\dif u}{\dif x} = \dfrac{Q}{\lambda} \\
            &\implies u(x) = \int \dfrac{Q(x)}{\lambda(x)} \difx
        \end{aligned}$

        综上,零解 + 特解
        $$y = C\exp\left[-\int P(x)\difx\right]\difx 
            \; + \; \exp\left[-\int P(x)\difx\right]\int Q(x)\exp\left[\int P(x)\difx\right]\difx$$

    \section{高阶}
        \subsection{可降阶}
        \begin{itemize}
            \item $\difk{y}{n} = f(x)$ \\
                $n$ 次积分。
            \item 不显含 $y$:$y'' = f(x, y')$ \\
                换元 $p = \dfrac{\dif y}{\dif x}$,化为一阶 $\dfrac{\dif p}{\dif x} = f(x, p)$,再积分得 $y$。
            \item 不显含 $x$:$y'' = f(y, y')$ \\
                设 $\dfrac{\dif y}{\dif x} = p(y)$ 不显含 $y$,
                则 $y'' = \dfrac{\dif p(y)}{x} = \dfrac{\dif p}{\dif y} \cdot \dfrac{\dif y}{\dif x} 
                        = p\dfrac{\dif p}{\dif y}$,\\
                化为 $p \cdot p' = f(y, p)$。
        \end{itemize}

        \subsection{常系数线性齐次方程}

        通解:
        \begin{itemize}
            \item $k$ 重实根 $\lambda$:\\
                $\exp(\lambda x), \, x\exp(\lambda x), \, \cdots, \, x^{k - 1}\exp(\lambda x)$
            \item $k$ 重共轭复根 $\alpha \pm \beta\imunit$:
                \begin{longtable}[h]{*{2}{w{c}{4cm}} w{c}{0.5cm} w{c}{4cm}}
                    $\exp(\alpha x)\cos(\beta x)$ & $\exp(\alpha x)\cdot x\cos(\beta x)$ 
                    & $\cdots$ & $\exp(\alpha x)\cdot x^{k - 1}\cos(\beta x)$ \\
                    $\exp(\alpha x)\sin(\beta x)$ & $\exp(\alpha x)\cdot x\sin(\beta x)$ 
                    & $\cdots$ & $\exp(\alpha x)\cdot x^{k - 1}\sin(\beta x)$
                \end{longtable}
        \end{itemize}

        特解:待定系数
        \begin{itemize}
            \item $F(x) = \varphi(x)\exp(\mu x)$,其中 $\varphi(x) \in \mathrm{Poly}$:\\
                $y^\ast = x^k P(x) \exp(\mu x)$,\\
                其中 $\mu$ 为 $k$ 重实根,$P(x) \in \mathrm{Poly}, \deg P(x) = \deg\varphi(x)$。
            \item $F(x) = \varphi(x)\exp(\mu x)\cos(\nu x) \, / \, \varphi(x)\exp(\mu x)\sin(\nu x)$,
                其中 $\varphi(x) \in \mathrm{Poly}$:\\
                $y^\ast = x^k\exp(\mu x)[P_1(x)\cos(\nu x) + P_2(x)\sin(\nu x)]$,\\
                其中 $\mu \pm \nu\imunit$ 为 $k$ 重共轭复根,
                $P_1(x),P_2(x) \in \mathrm{Poly}, \deg P_1(x) = \deg P_2(x) = \deg\varphi(x)$。
        \end{itemize}
\end{document}