\documentclass[12pt, a4paper, UTF8, openany]{ctexbook}

\usepackage{geometry}%设置页边距
\geometry{left=1.54cm, right=1.54cm, top=2.18cm, bottom=2.18cm}

\usepackage{tikz}%绘图

\usepackage{amsmath, amssymb}%数学

\usepackage{array}%表格基础设置
\usepackage{longtable}%跨页表格
\usepackage{booktabs}%表格水平线

\usepackage[colorlinks = true, 
            linkcolor = black]{hyperref}%超链接跳转

\usepackage{color}

\newcommand{\dif}{\mathrm{d}}
\newcommand{\difx}{\,\mathrm{d}x}

\newcommand{\difk}[2]{{#1}^{(#2)}}

\newcommand{\field}[1]{\mathbb{#1}}

\newcommand{\imunit}{\mathrm{i}}%虚数单位

\newcommand{\poly}{\mathrm{Poly}}

\newcommand{\mat}[1]{\mathbf{#1}}

\newcommand{\uzero}{\mathop{U}\limits^{\circ}}
\newcommand{\eps}{\varepsilon}

\linespread{1.05}%行间距
\setlength{\parskip}{8pt}%段落间距
\setlength{\parindent}{0pt}%不缩进

\title{数学分析}
\author{1001}

\begin{document}
    \maketitle

    %未知错误
    % \newpage
    \tableofcontents

    \chapter{函数,极限,连续}

    \section{基础}
    有界;有上/下确界 $\sup / \inf$。

    {\heiti 完备性}:有上/下界的非空实数集有上/下确界。

    单调:含等号。严格单调。

    邻域:$U(x_0, \eps) := \{x : |x - x_0| < \eps\}$。\\
    去心邻域:$\uzero(x_0, \eps) := U(x_0, \eps) \backslash \{x_0\}$。
    
    \section{极限}
        \subsection{定义}
        数列极限:$\forall \eps > 0$,$\exists N \in \field{N}_+$,使得 $\forall n > N$,$|a_n - A| < \eps$。

        函数极限:趋于无穷类似\\
        (Cauchy)$\eps-\delta$ 式:
        $\forall \eps > 0$,$\exists \delta > 0$,使得 $\forall x \in \uzero(x_0, \delta)$,$|f(x) - A| < \eps$。\\
        (Heine)序列式:$\forall x_n \to x_0$,$\lim_{n \to \infty} f(x_n) = A$。

        左极限 $x \to x_0^-$,右极限 $x \to x_0^+$。\\
        $\lim_{x \to x_0} = A \iff \lim_{x \to x_0^{-}} = \lim_{x \to x_0^{+}} = A$。

        \subsection{性质}
        极限唯一。\\
        证明:反证。取 $\eps = \frac{A + B}{2}$,即得矛盾。

        有界性:数列有界,函数局部有界。(上下均有界。)

        保序性:$\forall l_n \leqslant a_n \leqslant r_n$,则 $\lim l \leqslant \lim a \leqslant \lim r$。\\
        函数局部保序。\\
        推论:保号性。局部保号性。

        收敛数列的任意子列均收敛于 $A$。\\
        函数版本即 Heine 定理(序列式定义)。

        四则运算,函数复合。{\heiti 前提}:有定义。

        \subsection{判别与求值}
        单调有界准则。

        夹逼准则。

        归并原理:研究子列极限。数列、函数均可\\
        {\heiti 应用}:证明发散。

        闭区间套:$\{[a_n, b_n]\}$,满足 $[a_1, b_1] \supset [a_2, b_2] \supset \cdots$,且长度趋于零 $\lim b_n - a_n = 0$。\\
        闭区间套原理:存在{\heiti 唯一} $\xi$,使得 $\cap [a_i, b_i] = \{\xi\}$。证明:夹逼。

        Weierstrass 定理:有界数列必有收敛子列。\\
        证明:每次对分值域区间,总有一个子区间含有无穷多项。在另一子区间任取 $x_n$,在无穷多的子区间内递归。
        值域区间构成闭区间套,则构造了收敛子列 $\{x_n\}$。\\
        {\heiti 应用}:先构造极限值 $A$,再证明收敛于 $A$。

        Cauchy 数列 $\{a_n\}$:满足 $\forall \eps > 0$,$\exists N \in \field{N}_+$,
        使得 $\forall m, n > N$,$|a_m - a_n| < \eps$。\\
        等价叙述:$\forall n > N, p \in \field{N}_+$,$|a_n - a_{n + p}| < \eps$。

        Cauchy 收敛原理:\\
        a. $\{a_n\}$ 收敛的充要条件是 $\{a_n\}$ 为 Cauchy 列。证明:取 $\eps / 2$。\\
        b. $\lim_{x \to x_0} f(x)$ 存在的充要条件是 $\forall \eps > 0$,$\exists \delta > 0$,
        使得 $\forall x_1, x_2 \in \uzero(x_0, \delta)$,$|f(x_1) - f(x_2)| < \eps$。
        证明:充分性同数列;必要性需证极限均相同,反证,构造。

        {\heiti 递推式}:等式两端同时取极限,解方程。需证明 $A$ 存在。

        $$\lim_{x \to 0} \frac{\sin x}{x} = 1$$
        $$\lim_{x \to \infty} (1 + \frac{1}{x})^x = e$$

        \subsection{Stolz 定理}
        对一类 $\infty / \infty$ 型数列极限适用。

        若 $\{x_n\}$ 满足\\
        a. $\{x_n\}$ 严格递增,且恒正。\\
        b. $\lim x_n = +\infty$。\\
        则若 $\lim \dfrac{y_n - y_{n - 1}}{x_n - x_{n - 1}} = A \neq \infty$,有 $\lim \dfrac{y_n}{x_n} = A$。

    \section{无穷大小}
        \subsection{定义及性质}
        无穷小:$a_n \to 0$。\\
        无穷大:$a_n \to \infty$。(无界不等于无穷大,如振荡。)

        $\lim f(x) = A \; \iff \; f(x) = A + \alpha(x)$,其中 $\alpha(x)$ 为无穷小。\\
        {\heiti 应用}:利用无穷小经行显式的运算,舍弃无穷小即得极限。

        无穷大/小与其他数的四则运算:仅对有限和有定义。按常理。

        \subsection{阶}
        \begin{itemize}
            \item $\psi(x) = O(\varphi(x))$:$\frac{\psi(x)}{\varphi(x)}$ 有界。
            \item $\psi(x) = o(\varphi(x))$:$\frac{\psi(x)}{\varphi(x)}$ 是无穷小。
            \item $\psi(x) \sim \varphi(x)$:$\lim \frac{\psi(x)}{\varphi(x)} = 1$。这时称等价。\\
                此时有 $\psi(x) = \varphi(x) + o(\varphi(x))$。
            \item $\psi(x)$ 是 $\varphi(x)$ 的 $k$ 阶:$\lim \frac{\psi(x)}{\varphi^k(x)} = A \neq \pm \infty$。
        \end{itemize}
        $O, o$ 都是对 $\psi(x)$ 的限制($\psi(x) \leqslant k\varphi(x)$)。$o(\varphi(x)) = O(\varphi(x))$。

        高/低阶无穷大/小:按常理。

        \subsection{等价替换}
        等价无穷小:$x \to 0$ 时,\\
        $x \; \sim \; \sin x \; \sim \; \tan x \; \sim \; 
            \arcsin x \; \sim \; \arctan x \; \sim \; \exp(x) - 1 \; \sim \; \ln(1 + x)$ \\
        $1 - \cos^2 x \; \sim \; \frac{1}{2}x^2$ \\
        $(1 + x)^\alpha - 1 \; \sim \; \alpha x$

        带着高阶无穷小运算也可。本质是泰勒展开。

    \section{连续函数}
        \subsection{定义及间断点}
        在 $x_0$ 处连续:$\lim_{x \to x_0} f(x) = f(x_0)$。\\
        左/右连续。\\
        连续区间:端点单侧连续,内部处处连续。连续函数 $f \in C(a, b), C[a, b]$。

        $\text{间断点}
        \begin{cases}
            \text{第一类:左右极限存在}
            \begin{cases}
                \text{可去:左右相等}\\
                \text{跳跃:左右不等}
            \end{cases}\\
            \text{第二类:左右有极限不存在}
        \end{cases}$

        极限存在指 $A \neq \pm\infty$。间断点类型与 $f(x_0)$ 是否有定义及其具体值无关。\\
        第二类的特例:无穷,振荡。

        跳跃:$a^{1/x}$,在 $x = 0$ 附近。\\
        无穷:$1/x$,在 $x = 0$ 附近。

        \subsection{性质}
        连续函数的运算仍为连续函数,证明平凡。具体地:\\
        四则运算;复合函数;反函数;\\
        幂指函数:$f(x)^{g(x)} = \exp(g(x)\ln f(x))$。

        初等函数在定义区间上连续。

        {\heiti 应用}:对连续函数,$\lim$ 与函数符号 $f$ 可任意换序。$\lim f(x) = f(\lim x)$。

        \subsection{闭区间上连续函数}
        有界性。\\
        证明:反证。若无界,由 Weierstrass定理构造 $x_n \to x_0$,$\lim |f(x_n)| = +\infty$。
        但连续得 $\lim f(x_n) = f(\lim x_n) = f(x_0) < +\infty$。矛盾。

        最大最小值定理:能取到 $\max, \min$。\\
        证明:取上确界 $M = \sup f$。取 $\delta = 1/n$,必有 $x_n$ 满足 $|f(x_n) - M| = M - f(x_n) < \delta$。
        用 Weierstrass 定理和夹逼准则。

        零点存在定理:$f(a)f(b) < 0$,则存在 $f(\xi) = 0$。\\
        证明:构造闭区间套。

        介值定理:对 $f(a), f(b)$ 之间的任一值 $\mu$,存在 $f(\xi) = \mu$。\\
        推论:$\forall \min \leqslant \mu \max$,存在 $f(\xi) = \mu$。\\
        推论:值域 $I = \{f(x): x \in [a, b]\}$ 是闭区间。\\
        证明:零点存在定理。

        一致连续:$\forall \eps > 0$,$\exists \delta > 0$,
        使得 $\forall x_1, x_2, \, |x_1 - x_2| < \delta$,$|f(x_1) - f(x_2)| < \eps$。\\
        序列式表述:$\forall \{x_n\},\{y_n\}, \, \lim x_n - y_n = 0$,有 $\lim f(x_n) - f(y_n) = 0$。\\
        几何意义:没有越来越陡的地方。反例:$f(x) = 1 / x$ 在 $x = 0$ 附近。

        闭区间上连续函数一致连续。\\
        证明:反证。存在 $\eps > 0$,使得 $\forall \delta > 0$,总存在 $|x_1 - x_2| < \delta$,$|f(x_1) - f(x_2)| > \eps$。
        取 $\delta = 1 / n$,由此取出 $\{x_n\}, \{y_n\}$。由 Weierstrass 定理取出收敛子列 $\{x_n'\}$,取 $\{y_n'\}$ 为与之逐项对应。
        又由 $\lim x_n - y_n = 0$ 知 $\lim x_n' = \lim y_n'$,于是 $\lim f(x_n') - f(y_n') = f(\lim x_n') - f(\lim y_n') = 0$。
        这与 $|f(x_n') - f(y_n')| > \eps$ 矛盾。

        $f$ 在闭区间 $[a, b]$ 上一致连续的充要条件是,$f$ 在 $[a, b]$ 上处处连续。

    \chapter{一元函数微分}
    \section{定义}

    \section{求导}

    \section{微分中值定理}

    \section{L'Hôpital 法则}

    \section{Taylor 定理}
    多项式拟合函数,记 $$P_n(x) = \sum_{k = 0}^{n} \frac{\difk{f}{k}(x_0)}{k!}(x - x_0)^k$$
    $R_n(x) = f(x) - P_n(x)$。

    对 $P_n(x)$ 求 $k$ 阶导,则所有幂次 $t < k$ 的项为零,
    所有幂次 $t \geqslant k$ 的项形如 $\dfrac{\difk{f}{t}(x_0)}{(t - k)!}(x - x_0)^{t - k}$。
    幂次永远与阶乘相同。

    总有 $\forall k < n, 
    \difk{R_n}{k}(x_0) = \difk{f}{k}(x_0) - \difk{f}{k}(x_0) 
    - \sum_{t > k}\frac{\difk{f}{t}(x_0)}{(t - k)!}(x_0 - x_0)^{t - k} = 0$
    
        \subsection{误差估计}
        Peano 余项:$$R_n(x) = o((x - x_0)^n)$$
        证明:$f(x)$ 在 $x_0$ 处有 $n$ 阶导数,则只保证 $x_0$ 领域内有 $1\sim(n - 1)$ 阶连续导数,只能用 $n - 1$ 次洛必达法则。 \\
        $\begin{aligned}
            \lim_{x \to x_0} \frac{R_n(x)}{(x - x_0)^n} &= \mathop{\cdots}\limits^{\text{L'Hôpital}} = 
            \lim_{x \to x_0} \frac{\difk{R_n}{n - 1}(x)}{n!(x - x_0)} \\
            &= \frac{1}{n!}\lim_{x \to x_0} \frac{\difk{f}{n - 1}(x) - \difk{f}{n - 1}(x_0) - \difk{f}{n}(x_0)(x - x_0)}{x - x_0} \\
            &= \frac{1}{n!} \lim_{x \to x_0} 
                \left[\frac{\difk{f}{n - 1}(x) - \difk{f}{n - 1}{x_0}}{x - x_0} - \difk{f}{n}{x_0}\right]
        \end{aligned}$\\
        最后一个等号源自 $f^{n}(x_0)$ 的定义。

        Lagrange 余项:
        $$R_n(x) = \frac{\difk{f}{n + 1}(\xi)}{(n + 1)!}(x - x_0)^{n + 1}, \; \xi \in x \sim x_0$$
        证明:
        $R_n(x) = \dfrac{\difk{f}{n + 1}(\xi)}{(n + 1)!}(x - x_0)^{n + 1} 
        \iff \dfrac{R_n(x)}{(x - x_0)^{n + 1}} = \dfrac{\difk{f}{n + 1}(\xi)}{(n + 1)!}$ \\
        连用 $n + 1$ 次 Cauchy 中值定理,每次从区间内取 $\xi_k$。\\
        $\begin{aligned}
            \dfrac{R_n(x)}{(x - x_0)^{n + 1}} &= \dfrac{R_n(x) - R_n(x_0)}{(x - x_0)^{n + 1} - (x_0 - x_0)^{n + 1}} = 
            \dfrac{\difk{R_n}{1}(\xi_1)}{\difk{((x - x_0)^n)}{1}}
            = \mathop{\cdots}\limits^{\text{如法炮制}} \\
            &= \dfrac{\difk{R_n}{n}(\xi_n) - \difk{R_n}{n}(x_0)}{(n + 1)!(x - x_0) - (x_0 - x_0)} = 
            \dfrac{\difk{f}{n + 1}(\xi)}{(n + 1)!}
        \end{aligned}$

        改写 Lagrange 余项(注意 $\theta$ 取值为开区间,可取端点):
        $$R_n(x) = \dfrac{\difk{f}{n + 1}\left[x_0 + \theta(x - x_0)\right]}{(n + 1)!}(x - x_0)^{n + 1}, \; \theta \in (0, 1)$$

        积分余项(注意阶乘为 $n$ 而非 $n + 1$):
        $$R_n(x) = \dfrac{\int_{0}^{1} (1 - t)^n\difk{f}{n + 1}(x_0 + t(x - x_0))\,\dif t}{n!} (x - x_0)^{n + 1}$$

        Cauchy 余项(注意 $\theta$ 取值为闭区间,可能仅能取端点):
        $$R_n(x) = \dfrac{(1 - \theta)^n\difk{f}{n + 1}(x_0 + \theta(x - x_0))}{n!}(x - x_0)^{n + 1}, \; \theta \in [0, 1]$$
        证明:对积分余项用积分中值定理。

        \subsection{Maclaurin 公式}
        $x_0 = 0, \, R_n(x) = \dfrac{\difk{f}{n + 1}(\theta x)}{(n + 1)!}x^{n + 1}$

        \begin{longtable}[h]{w{l}{12cm}}
            $\exp x = 1 + x + \dfrac{x^2}{2} + \dfrac{x^3}{3!} + \cdots + \dfrac{x^n}{n!} + \cdots$ \\
            \addlinespace
            $\ln(1 + x) = x - \dfrac{x^2}{2} + \dfrac{x^3}{3} - \dfrac{x^4}{4} 
                + \cdots + (-1)^{k + 1}\dfrac{x^k}{k} + \cdots$ \\
            \midrule
            $\sin x = x - \dfrac{x^3}{3!} + \dfrac{x^5}{5!} - \dfrac{x^7}{7!} 
                        + \cdots + (-1)^{k}\dfrac{x^{2k + 1}}{(2k + 1)!} + \cdots$ \\
            \addlinespace
            $\cos x = 1 - \dfrac{x^2}{2!} + \dfrac{x^4}{4!} - \dfrac{x^6}{6!}
                        + \cdots + (-1)^{k}\dfrac{x^{2k}}{(2k)!} + \cdots$ \\
            \midrule
            $(1 + x)^\alpha = 1 + \alpha x + \dbinom{\alpha}{2}x^2 + \dbinom{\alpha}{3}x^3 
                + \cdots + \dbinom{\alpha}{k}x^k + \cdots$ \\
            \midrule
            $\begin{aligned}
                \dfrac{1}{1 + x} &= \dfrac{\dif}{\dif x}\ln(1 + x) \\
                    &= 1 - x + x^2 - x^3 + \cdots + (-1)^kx^k + \cdots
            \end{aligned}$ \\
            $\dfrac{1}{1 - x} = 1 + x + x^2 + x^3 + \cdots + x^k + \cdots$ \\
            \addlinespace
            $\sqrt{1 + x} = 1 + \dfrac{1}{2}x - \dfrac{1}{8}x^2 + \cdots 
                + (-1)^{k + 1}\dfrac{(2k - 3)!!}{(2k)!!}x^k + \cdots$ \\
            \addlinespace
            $\dfrac{1}{\sqrt{1 + x}} = 1 - \dfrac{1}{2}x + \dfrac{3}{8}x^2 
                - \cdots + (-1)^k\dfrac{(2n - 1)!!}{(2n)!!}x^n + \cdots$
        \end{longtable}
        
        \subsection{应用}
        定量估计误差:Lagrange 余项。

        近似计算。

        求极限 / 高阶导数值:展开成多项式。

        函数比大小:Lagrange 余项。适度展开,计算点值,etc。

    \section{函数性质研究}
        \subsection{单调性}

        \subsection{极值与最值}
        驻点:$f'(x_0) = 0$ 的点 $x = x_0$。

        判定:
        \begin{itemize}
            \item $f'(x)$ 变号:\\
                a. 由负到正:极小值;\\
                b. 由正到负:极大值。
            \item $f'(x_0) = 0$,利用 $f''(x_0)$:\\
                a. $f''(x_0) > 0$:极小值;\\
                b. $f''(x_0) < 0$:极大值。
            \item 推广:$\difk{f}{1}(x_0) = \cdots = \difk{f}{k - 1}(x_0) = 0, \, \difk{f}{k}(x_0) \neq 0$:\\
                {\heiti 当且仅当 $k$ 为偶数}时,为极值;判定同二阶导。
        \end{itemize}
        证明:泰勒展开,Peano 余项即可。

        极值点还可能包括{\heiti 不可导点},但{\heiti 边界点依定义不是}。

        最值点:极值点和边界点。

        \subsection{凹凸性}
        {\heiti 上凸 = 凸} = 下凹:从上/下方观察,曲线向 $y$ 轴正半轴方向凸出。

        {\heiti 上凹 = 凹} = 下凸:从上/下方观察,曲线向 $y$ 轴负半轴方向凸出。

        上凹的等价定义(上凸函数取反即得上凹函数):
        \begin{itemize}
            \item $\forall P_1, P_2$,线段 $\overline{P_1P_2}$ 在弧 $\overset{\frown}{P_1P_2}$ 上方。
            \item $\forall x_0, (x_0, f(x_0))$ 处切线在曲线下方。
            \item $\forall x_1, x_2 \; f(\frac{x_1 + x_2}{2}) < \frac{1}{2}(f(x_1) + f(x_2))$。
            \item $\forall x_1, x_2, \lambda \in (0, 1) \; 
                f(\lambda x_1 + (1 - \lambda)x_2) < \lambda f(x_1) + (1 - \lambda)f(x_2)$。
        \end{itemize}

        可证明凸函数除了端点以外,为连续函数。

        凹凸性判定:$\forall x \in I$:\\
        a. $\difk{f}{2}(x) > 0$:上凹。 \\
        b. $\difk{f}{2}(x) < 0$:上凸。 \\
        证明:依据中点凸定义,在中点处泰勒展开,Peano 余项即可。
        
        拐点:凹凸性改变的点 $(x, y)$,$x$ 是{\heiti 一阶导极值点}。\\
        拐点判定:$\difk{f}{2}(x) = 0$,必要{\heiti 不充分}。见极值点判定。

        \subsection{渐近线}
        水平 / 铅直:趋于 $\pm \infty$ / 奇点。

        斜:$$\begin{aligned}
            k &= \lim_{x \to \infty} \frac{f(x)}{x} \\
            b &= \lim_{x \to \infty} (f(x) - kx)
        \end{aligned}$$

        \subsection{曲率}
        $$k := \lim_{\Delta s \to 0} \left| \frac{\Delta\alpha}{\Delta s} \right| = \left|\frac{\dif\alpha}{\dif s}\right|$$

        对于 $y = y(x)$,则 $\alpha = \alpha(x), s = s(x)$。具体地:

        $\alpha$:
        $\tan\alpha = y' \implies \alpha = \arctan y' \implies \dfrac{\dif\alpha}{\dif x} = \dfrac{y''}{1 + (y')^2}$。

        $s$:
        $\dif s = \sqrt{(\dif x)^2 + (\dif y)^2} \implies \dfrac{\dif s}{\dif x} = \sqrt{1 + (y')^2}$。

        $$k = \left|\frac{\dif\alpha}{\dif s}\right| = \left| \frac{y''}{\left( 1 + (y')^2 \right)^{\frac{3}{2}}} \right|$$

        曲率半径 $\rho := \frac{1}{k}$。\\
        曲率圆:圆心在曲线凹向,与曲线相切(即共切线)。

        \subsection{作图}

    \chapter{一元函数积分}
    \section{常义}
        \subsection{定义}
        $$\int_{a}^{b} f(x)\dif x := \lim\limits_{d = \max\{\Delta x\} \to 0} \sum f(\xi_i)\Delta x_i$$

        几何意义:面积代数和。定义:$\int_{b}^{a} := -\int_{a}^{b},\; \int_{a}^{a} := 0$

        记 $M_k = \sup\{f(x): x \in [x_{k - 1}, x_k]\}, \; m_k = \inf\{f(x): x \in [x_{k - 1}, x_k]\}$。
        定义 Darboux 大和 $S_n = \sum M_k\Delta x_k$,Darboux 小和 $s_n = \sum m_k\Delta x_k$。
        则 Riemann 可积的充要条件是 $\lim_{d \to 0} S_n - s_n = 0$。

        注:积分变量任意。$\int_{a}^{x} f(t)\dif t = \int_{a}^{x} f(x)\dif x$。

        \subsection{可积}
        \begin{itemize}
            \item $f\in C[a, b]$。\\
            证明:连续则{\heiti 一致连续},则存在 $d$ 使 $\forall |u - v| < d, |f(u) - f(v)| < \epsilon$。
            \item {\heiti 有界}函数 $f$ 在 $[a, b]$ 有有限个第一类间断点。
            \item {\heiti 有界}函数 $f$ 在 $[a, b]$ 单调。
        \end{itemize}
        
        \subsection{性质}
        \begin{itemize}
            \item 线性:$\int_{a}^{b} (\lambda f(x) + \mu g(x))\dif x = \lambda\int_{a}^{b}f(x)\dif x + \mu\int_{a}^{b}g(x)\dif x$。
            \item 区间可加:$\int_{a}^{b} = \int_{a}^{c} + \int_{c}^{b}$。
            \item 保序:$f(x) \leqslant g(x) \; \implies \; \int_{a}^{b} f(x)\dif x \leqslant \int_{a}^{b} g(x)\dif x$。\\
                推论:$m(b - a) \leqslant \int_{a}^{b} f(x)\dif x \leqslant M(b - a)$;
                $|\int_{a}^{b} f(x)\dif x| \leqslant \int_{a}^{b} |f(x)|\dif x$。\\
                {\heiti 应用}:估值。
            \item 积分中值定理:$f(x)$ {\heiti 连续},$g(x)$ 不变号,
                则 $\exists \xi \in [a, b], \; \int_{a}^{b} f(x)g(x)\dif x = f(\xi)\int_{a}^{b} g(x)\dif x$。\\
                证明:不妨 $g \geqslant 0$,$m\int_{a}^{b} g \leqslant \int_{a}^{b} fg \leqslant M\int_{a}^{b} g 
                \implies m \leqslant (\int_{a}^{b} fg) / (\int_{a}^{b} g) \leqslant M$。由拉格朗日中值即得。\\
                推论:$\exists \xi, \; \int_{a}^{b} f(x)\dif x = f(\xi)(b - a)$。
        \end{itemize}

        \subsection{原理}
        \begin{itemize}
            \item Newton - Leibniz 公式:$$\int_{a}^{b} f(x)\dif x = F(x)|_a^b$$\\
                证明:$F(b) - F(a) = \sum F(x_k) - F(x_{k - 1}) = \sum F'(\xi_k)\Delta x_k 
                    = \sum f(\xi_k)\Delta x_k = \int_{a}^{b} f(x)\dif x$。
        \end{itemize}
        
        积分上限函数:$\Phi(x) = \int_{a}^{x} f(t)\dif t$。基本定理:
        \begin{itemize}
            \item 第一:$f \in C[a, b]$,则 $$\Phi(x)' = \frac{\dif}{\dif x}\int_{a}^{x} f(t)\dif t = f(x)$$
                {\heiti 说明连续函数必有原函数}。\\
                证明:$\Delta\Phi(x) = \int_{x}^{x + \Delta x} f(t)\dif t = f(\xi)\Delta x$。
            \item 第二:$F + C$ 为所有原函数。
        \end{itemize}

        {\heiti 积分上限求导}:
        $$\Psi(x) = \int_{u(x)}^{v(x)} f(t)\dif t = -\int_{p}^{u(x)} + \int_{p}^{v(x)} 
        \implies \frac{\dif}{\dif x}\Psi(x) = -f(u(x))u'(x) + f(v(x))v'(x)$$ 
        为复合函数 $\Phi(u/v)$。若 $f(t)$ 为 $f(x, t)$,应先拆出 $x$。
        
    \section{计算}
        \subsection{积分法}
        \begin{itemize}
            \item 换元:$$\dif F(\phi(x)) \iff F'(\phi(x))\,\dif\phi(x)$$
                左到右:换元 $x = u(x)$,并且 $u(x)$ {\heiti 单调};右到左:凑微分。\\
                定积分换元:换变量同时换上下限。
            \item 分布:$$\dif uv = v\dif u + u\dif v \; \implies \; \int v\dif u = uv - \int u\dif v$$
                定积分分布:相同的上下限 $|_a^b,\, \int_{a}^{b}$。
        \end{itemize}

        \subsection{幂函数}
        $$\int x^\alpha\,\dif x = 
            \begin{cases}
                \frac{1}{\alpha + 1}\, x^{\alpha + 1} + C &\alpha\neq -1 \\
                \ln|x| + C &\alpha = -1
            \end{cases}$$

        \subsection{指数与对数}
        $$\int \alpha^x\,\dif x = \frac{\alpha^x}{\ln\alpha} + C \quad \alpha > 0 \land \alpha \neq 1$$\\
        特别地,$$\int e^x\,\dif x = e^x + C$$

        $$\int\ln x\difx = x\ln x - x + C$$

        $e^x$ 与常数的分式:强行提出 $e^x$ 使 $\difx \to \dif t := \dif e^x$。换元化为有理函数。

        $\int e^x(f(x) + f'(x))\difx = e^xf(x) + C$。

        \subsection{(反)三角函数}
        \begin{itemize}
            \item 基础
                \begin{longtable}[h]{*{2}{w{l}{8cm}}}
                    $\int \sin x\,\dif x = -\cos x + C$ & $\int \cos x\,\dif x = \sin x + C$\\
                    &\\
                    $\int \cot x\,\dif x = -\ln|\sin x| + C$ & $\int \tan x\,\dif x = -\ln|\cos x| + C$\\
                    $\int \csc^2 x\difx = -\cot x + C$ & $\int \sec^2 x\difx = \tan x + C$\\
                    &\\
                    $\int \csc x\difx = -\ln|\csc x + \cot x| + C$ & $\int \sec x\difx = \ln|\sec x + \tan x| + C$\\
                    $\int \csc x \cot x\difx = -\csc x + C$ & $\int \sec x \tan x\difx = \sec x + C$\\
                    &\\
                    $\int \frac{\dif x}{\sqrt{1 - x^2}} = \arcsin x + C = -\arccos x + C$ & 
                    $\int \frac{\dif x}{1 + x^2} = \arctan x + C$
                \end{longtable}
                证明:$\int \tan$:换元 $\sin x\difx = -\dif\cos x$。\\
                    $\int \sec$:$(\sec + \tan)' = \sec\tan + \sec^2 = \sec(\sec + \tan) 
                                \implies \sec = \frac{(\sec + \tan)'}{\sec + \tan}$\\
                    或:$\int\sec\difx = \int \frac{\cos x}{\cos^2 x}\dif x = \int \frac{\dif\sin x}{1 - \sin^2 x}$
            
            \item 万能代换
                $$\begin{aligned}
                    t &= \tan\frac{x}{2}\quad &\dif x &= \frac{2}{1 + t^2}\dif t\\
                    \sin x &= \frac{2t}{1 + t^2}\quad &\cos x &= \frac{1 - t^2}{1 + t^2}
                \end{aligned}$$\\
                证明:$x = 2\arccos t$;二倍角公式展开,除以 $1 = \sin^2 x + \cos^2 x$,同除 $\cos^2 x$。

            \item 分式\\
                分母齐次:提 $\cos^2 x$,凑 $\dif x \to \dif\tan x$。利用二倍角凑出二次式以便换元。\\
                $\int \dfrac{a\cos x + b\sin x}{c\cos x + d\sin x}\dif x$:
                    拆分子为 $\lambda(c\cos x + d\sin x) + \mu(c\cos x + d\sin x)'$。
        \end{itemize}

        \subsection{换元}
        $\dfrac{1}{a^2 + x^2}, \dfrac{1}{\sqrt{a^2 - x^2}} \; \implies \; \arctan, \arcsin$。

        $\sqrt[n]{ax + b}$:令 $t = \sqrt[n]{ax + b}$,反解 $x$。

        $\sqrt{x^2 \pm a^2}, \sqrt{a^2 \pm x^2}$:构造直角三角形,其中一边为 $a$。换元 $\theta$,以三角函数表示其他边。\\
        二次式:配方化为上述。

        $(x \pm \frac{1}{x})' = 1 \pm \frac{1}{x^2}$,除以 $x^2$ 构造 
        $(x \pm \frac{1}{x}),\, (x \pm \frac{1}{x})^2, \, (1 \pm \frac{1}{x^2}), \, \mathrm{etc.}$

        \subsection{分布}
        优先级:反函数 $\to$ 对数 $\to$ 幂函数 $\to$ 三角函数 $\to$ 指数函数\\
        在 $u\dif v$ 中,前面的优先视为 $u$ 来求导。

        间接积分:
        \begin{itemize}
            \item $\sin, \cos$:多次求导会恢复,继而可列方程求解(如果积分存在)。\\
                此外,等式两侧同时消去 $\int f(x)\difx$ 会留下一个常数 $C$,本质是函数族。
            \item 递归:构造同构
                $$\begin{aligned}
                    I_n &= \int \tan^n\difx = \int \tan^{n - 2}x (\sec^2 - 1)\difx
                        = \int \tan^{n - 2}x\,\dif\tan x - \int\tan^{n - 2}x\difx \\
                    \implies &I_n = \frac{1}{n - 1}\tan^{n - 1}x - I_{n - 2}
                \end{aligned}$$
                
                $$\begin{aligned}
                    J_n &= \int_{0}^{\pi / 2} \sin^n\difx = \int_{0}^{\pi / 2} \cos^n\difx \text{ 证明:换元。} \\
                    J_n &= \int_{0}^{\pi / 2} \sin^{n - 2}x (1 - \cos^2 x)\difx 
                        = J_{n - 2} - \int_{0}^{\pi / 2} \sin^{n - 2}x \cos x \,\dif\sin x \\
                        &= J_{n - 2} - \int_{0}^{\pi / 2} \cos x \,\dif \frac{1}{n - 1}\sin^{n - 1} x \\
                        &= J_{n - 2} - \frac{1}{n - 1}\cos x \sin^{n - 1} x |_{0}^{\pi / 2} 
                            - \frac{1}{n - 1}\int_{0}^{\pi / 2} \sin^n x\difx \\
                    \implies &J_n = \frac{n - 1}{n} J_{n - 2}\\
                    \implies &J_n = \frac{(n - 1)!!}{n!!} J_{n \bmod 2} \quad (J_0 = \pi / 2, \; J_1 = 1)
                \end{aligned}$$
        \end{itemize}

        \subsection{有理函数}
        有理真分式 $R(x) = P(x) / Q(x)$,不妨 $Q(x)$ 为首一多项式,\\
        在 $\field{F} = \field{R}$ 分解分母得 $Q(x) = \prod (x - a)^{k} \, \prod (x^2 + px + q)^{r}$。
        其中二次式对应一对共轭复根。\\
        则 $R(x)$ 总能分解为 $$R(x) = \sum\sum_{t = 1}^k \frac{A}{(x - a)^t} + \sum\sum_{t = 1}^r \frac{Ax + B}{(x^2 + px + q)^t}$$
        分式个数至多与 $Q(x)$ 的因式个数相同。

        证明:先按不同的根分类,再对每个根证明可拆分为一串分母幂次递增的分式和。数论在多项式的推广作为第一步的基础。
        \begin{itemize}
            \item 引理之引理:$f_{1 \sim n} \in \poly$,记 $S = \{\sum u_i f_i : u_{1 \sim n} \in \poly\}$
            则存在 $g \in \poly$,使得 $S = g\poly$。\\
                证明:任取 $g$ 为 $S$ 中一个次数最低的多项式,设 $g = \sum u_i f_i$。用 $g$ 除每个 $f_i$:$f_i = p g + r$。\\
                $r = f_i - pg$ 展开仍为 $f_{1 \sim n}$ 的组合。若 $\deg r > -\infty$,
                则 $r \in S$,而除法约束 $\deg r < \deg g$,则矛盾。\\
                故 $r = 0$,于是 $g | f_i$ 对所有 $f_i$ 均成立。
            \item 引理 1:$f_{1 \sim n} \in \poly$,记 $g$ 为 $f_{1 \sim n}$ 的首一最大公因子,
            则 $\exists u_{1 \sim n} \in \poly$,使得 $\sum u_i f_i = g$。\\
                证明:由引理知 $\exists g$,$\forall f_i, \, g | f_i$。
                又有 $\forall f_i, \; d | f_i \; \iff \; \forall x \in S, \; d | x \; \iff \; d | g$,
                足以说明 $g$ 为 $\gcd$。\\
                $g \in S$ 说明 $u_{1 \sim n}$ 的存在性。
            \item 引理 2:不妨 $q \in \poly, \; q \neq 0$。对任意 $f \in \poly$,
            存在 $\{u_{n - 1}\}$ 满足 $\forall \deg u_i < \deg q$,
            并且 $f = \sum_{i = 0}^{n - 1} u_i q^i$。\\
                证明:通俗地讲,这就是 $f$ 在 $q$ 进制下展开。不断做除法取余“数”,即得 $u$。\\
                {\heiti 推论}:同除 $q^n$,
                则 $\dfrac{f}{q^n} = \sum_{i = 0}^{n - 1} \dfrac{u_i}{q^{n - i}} = \sum_{i = 1}^{n} \dfrac{v_i}{q^i}$。
            \item 证明:设 $Q(x) = \prod q_i^{k_i}$ 为因式分解,其中 $\deg q_i = 1/2$(实根或共轭复根)。
                记 $f_i = q_i^{k_i}$ 合并了所有重根,再记 $h_i = Q(x) / f_i(x)$。\\
                显然 $h_i = \prod_{j \neq i} q_j^{k_j}$,所有 $\{h_n\}$ 没有公因式,即 $g = 1$。
                由引理 1,存在 $\{u_n\}$ 使 $P(x) = g P(x) = \sum u_i h_i = \sum u_i Q / f_i$ 成立。\\
                同除 $Q(x)$,得 $\dfrac{P(x)}{Q(x)} = \sum \dfrac{u_i}{f_i} = \sum \dfrac{u_i}{q_i^{k_i}}$。\\
                对其中的每一项,由引理 2 可以展开,至此完成了全部论述。
        \end{itemize}

        为求分解,可待定系数,或凑。多项式除法。

        对每一类分别积分即可:
        \begin{itemize}
            \item $\int \frac{\dif x}{(x - a)^t}$:同幂函数。
            \item $\int \frac{Ax + B}{(x^2 + px + q)^t}$:\\
                先拆分:$Ax + B = \lambda(2x + p) + \mu = \lambda(x^2 + px + q)' + \mu$,
                对二次式整体换元,为幂函数。\\
                复根则有 $\Delta < 0$,故配方得 $x^2 + px + q = (x + \phi)^2 + v^2$。
                换元 $u = x + \phi$。\\
                后者化为 $I_t = \int\frac{\dif u}{(u^2 + v^2)^t}$。

                $$\begin{aligned}
                    I_n &= \frac{u}{(u^2 + v^2)^n} - \int u \frac{-n \cdot 2u}{(u^2 + v^2)^{n + 1}}\dif u
                        = \frac{u}{(u^2 + v^2)^n} + 2n\int \frac{u^2}{(u^2 + v^2)^{n + 1}}\dif u\\
                        &= \frac{u}{(u^2 + v^2)^n} + 2n\int \frac{(u^2 + v^2) - v^2}{(u^2 + v^2)^{n + 1}}\dif u\\
                        &= \frac{u}{(u^2 + v^2)^n} + 2nI_n - 2nv^2I_{n + 1}\\
                    \implies &I_{n + 1} = \frac{u}{2nv^2(u^2 + v^2)^n} + \frac{2n - 1}{2nv^2}I_n 
                    \quad I_1 = \frac{1}{v}\arctan\frac{u}{v} + C
                \end{aligned}$$
        \end{itemize}

        \subsection{特殊函数}
        对称区间上奇偶函数定积分:显然。注意需要为{\heiti 常义积分}。

        周期函数:一个周期内定积分不变。证明:换元使整体平移。

        复杂三角函数:先限制到一个周期,证明一个周期的积分为 0。\\
        方向:$f(x) + f(-x) = 0, \; f(x) + f(x + \pi) = 0$

    \section{广义积分}
        \subsection{定义}
        无穷积分:$$\int_{a}^{+\infty} f(x)\difx := \lim_{b \to +\infty} \int_{a}^{b} f(x)\difx$$
        若存在{\heiti 有穷}极限则称原式收敛,否则发散。\\
        {\heiti 无上下界积分}:
        $\exists t, \; \int_{-\infty}^{+\infty} = \int_{t}^{+\infty} + \int_{-\infty}^{t}$ {\heiti 均收敛}则称收敛。
        易知收敛时,$t$ 的选取没有任何影响。

        瑕积分:在 $x = a$ 附近无界 
        $$\int_{a}^{b} f(x)\difx := \lim_{\varepsilon \to 0^+} \int_{a + \varepsilon}^{b} f(x)\difx$$
        若存在{\heiti 有穷}极限则称原式收敛,否则发散。\\
        {\heiti 区间内的瑕点积分}:
        $\int_a^b = \lim_{\varepsilon \to 0^-} \int_{a}^{c + \varepsilon} + \lim_{\delta \to 0^+} \int_{c + \delta}^b$ 
        均收敛则称收敛。{\heiti 两个极限过程独立}。

        瑕积分可转化为无穷积分,换元 $a + \frac{1}{t}$。

        \subsection{性质}
        对定积分成立的性质均成立。

        Newton - Leibniz 公式:问题归为原函数在无穷远处 / 瑕点处是否收敛 / 有单侧极限。

        \subsection{审敛法}
        准则对任意一种广义积分均成立。

        比较准则 I:$0 \leqslant f(x) \leqslant g(x)$,有:\\
        a. $\int g(x)\difx$ 收敛,则 $\int f(x)\difx$ 收敛;\\
        b. $\int f(x)\difx$ 发散,则 $\int g(x)\difx$ 发散。

        比较准则 II:$f(x) \geqslant 0, \, g(x) > 0$,记趋于远处/瑕点时 $\lim \frac{f(x)}{g(x)} = \lambda$,有:\\
        a. $\lambda > 0$:$\int f(x)\difx$ 和 $\int g(x)\difx$ 同时收敛或同时发散;\\
        b. $\lambda = 0$:(说明 $f << g$,)$\int g(x)\difx$ 收敛则 $\int f(x)\difx$ 收敛;\\
        c. $\lambda = +\infty$:(说明 $f >> g$),$\int g(x)\difx$ 发散则 $\int f(x)\difx$ 发散。\\
        (总之,用分母的 $g$ 的收敛性判定分子的 $f$。二者的积分上限函数均单增,故有上界则有极限。)

        绝对收敛准则:若 $\int |f(x)|\difx$ 收敛,则 $\int f(x)\difx$ 收敛,称绝对收敛。\\
        若 $\int |f(x)|\difx$ 发散,但 $\int f(x)\difx$ 收敛,称条件收敛。\\
        证明:$0 \leqslant f(x) + |f(x)| \leqslant 2|f(x)|$,由极限的四则运算知收敛。

        实践上,取 $g(x) = 1 / x^p$。$p = 1$ 恒发散,不同积分的敛散性相反。
        \begin{longtable}[h]{w{c}{2cm}*{2}{|w{c}{6cm}}}
            $\int_I \frac{\dif x}{x^p}$ & 瑕积分 $I = (0, a]$ & 无穷积分 $I = [a, \infty)$ \\
            \midrule
            $0 < p \leqslant 1$         & 发散                & 收敛 \\
            \midrule
            $1 < p$                     & 收敛                & 发散 
        \end{longtable}

        比较 $f(x)$ 与 $\frac{C}{x^p} / \frac{C}{(x - a)^p}$ 或讨论 $\lim x^pf(x) / (x - a)^pf(x)$。
        
    \section{应用}
    \begin{itemize}
        \item 级数求极限:除 $n$ 使 $\sum \Delta = C$。
        \item 弧长:$\dif l = \sqrt{(\dif x)^2 + (\dif y)^2}$\\
            极坐标:换元 $x = \rho\cos\theta, y = \rho\sin\theta$,求导按乘积求导,\\
                有 $\dif l = \sqrt{[\rho'(\theta)]^2 + [\rho(\theta)]^2}\,\dif\theta$。
        \item 面积:\\
            极坐标:扇形近似 $\rho = \rho(\theta), \, \dif S = \frac{1}{2} \rho^2 \,\dif\theta$。
        \item 体积:\\
            层切 $\dif V = \pi r^2 \,\dif h$;柱剥:$\dif V = 2\pi h \,\dif r$。
        \item 含 $\int$ 的函数方程:求导,化为一般或微分方程。\\
            {\heiti 积分上下界相同}时为零,是一个{\heiti 初值}。
    \end{itemize}

    \chapter{常微分方程}
    \section{定义}
    常微分方程:$y = y(x)$,唯一的自变量是 $x$。

    阶:方程中的最高阶导数。e.g. $y'' + x^3y = 0$ 二阶。

    次:因变量或其导数的最高幂次。e.g. $y' + x^3y^2 + x^5 = 0$ 一阶二次。

    齐次:等式右侧是否为 $0$。区分像是零空间还是零空间的平移,与“次”的概念无关。\\
    齐次在多项式中又有次数相同义,依上下文区分。

    解:代入 $y = y(x)$ 得到恒等式。

    通解:含有任意常数的解,且相互独立的常数个数恰为阶数。相当于一个解的集合。\\
    e.g. $\dif^n y / \dif x^n = f(x)$,连续 $n$ 次积分。
    通解形如 $y(x) = y_0(x) + \mathrm{Poly}$,$n$ 次多项式的低 $n - 1$ 位在方程中均被抹去。\\
    {\heiti 通解未必包含全部解}。部分解为单点,产生自分母恒为零等平凡情况。

    特解:不含任意常数。相当于解集的一个元素。

    \section{一阶及本质一阶}
        \subsection{一阶线性}
        形式:$$\frac{\dif y}{\dif x} + P(x)y = Q(x)$$

        先齐次:$\dfrac{\dif y}{\dif x} + P(x)y = 0 \implies \dfrac{\dif y}{y} = -P(x)\difx 
        \implies y = C\exp\left[-\int P(x)\difx\right]$

        再非齐次:记 $\lambda = \exp\left[-\int P(x)\difx\right]$,$y = u(x)\lambda(x)$ 带入方程:\\
        $u'\exp - P\lambda u + Pu\lambda = Q \implies \dfrac{\dif u}{\dif x} = \dfrac{Q}{\lambda} 
            \implies u(x) = \int \frac{Q(x)}{\lambda(x)} \difx$

        综上,零解 + 特解
        $$y = C\exp\left[-\int P(x)\difx\right]\difx 
            \; + \; \exp\left[-\int P(x)\difx\right]\int Q(x)\exp\left[\int P(x)\difx\right]\difx$$
        
        \subsection{分离变量}
        形式:$$\frac{\dif y}{\dif x} = f(x)g(y) \implies \int \frac{\dif y}{g(y)} = \int f(x)\difx$$

        一般求通解不考虑分母为零等平凡情况。但这些平凡情况可能不能被通解表示。\\
        绝对值包含在 $C$ 中:$\ln|y| \to C\ln y$。

        化为分离变量:{\heiti 换元}。特别地,{\heiti 齐次}方程可换元。
        \begin{itemize}
            \item 齐次整式方程:尝试同除构造 $f(\dfrac{y}{x})$。
            \item  $\dfrac{\dif y}{\dif x} = f(\dfrac{y}{x})$:\\
                换元 $t = \dfrac{y}{x}$,则 $\dif y = \dif(tx) = t\difx + x\,\dif t$。\\
                $\implies x\dfrac{\dif t}{\dif x} + t = f(t) \implies \dfrac{\dif t}{f(t) - t} = \dfrac{\dif x}{x}$。
            \item $\dfrac{\dif y}{\dif x} = f(\dfrac{ax + by + c}{Ax + By +C})$:\\
                考虑作换元(平移)$z = x + x_0$,$w =  y + y_0$,希望消去 $c, C$ 使之同上。\\
                解 $\begin{cases}
                    ax_0 + by_0 + c = 0 \\
                    Ax_0 + By_0 + C = 0
                \end{cases}$\\
                默认不可约,故不会无穷多解。有唯一解,则已完成。\\
                无解,说明为两条平行线。记分子为 $u$,则分式形如 $\dfrac{u}{\lambda u + C'}$,以 $u$ 换 $y$ 可分离变量。
        \end{itemize}

        \subsection{可降阶}
        \begin{itemize}
            \item $\difk{y}{n} = f(x)$ \\
                $n$ 次积分。
            \item 不显含 $y$:$y'' = f(x, y')$ \\
                换元 $p = \dfrac{\dif y}{\dif x}$,化为一阶 $\dfrac{\dif p}{\dif x} = f(x, p)$,再积分得 $y$。
            \item 不显含 $x$:$y'' = f(y, y')$ \\
                设 $\dfrac{\dif y}{\dif x} = p(y)$ 不显含 $y$,
                则 $y'' = \dfrac{\dif p(y)}{x} = \dfrac{\dif p}{\dif y} \cdot \dfrac{\dif y}{\dif x} 
                        = p\dfrac{\dif p}{\dif y}$,\\
                化为 $p \cdot p' = f(y, p)$。
            \item Bernoulli 方程:$$\frac{\dif y}{\dif x} + P(x)y = Q(x)y^\alpha$$
                同除 $y^\alpha$,换元 $u = y^{1 - \alpha}$。\\
                $P(x)y \implies P(x)u$,
                $y^{-\alpha}\,\dif y \implies \frac{1}{1 - \alpha}\dif u$。
        \end{itemize}

    \section{高阶线性方程}
        \subsection{解的结构}
        线性方程:未知数及各阶导数均为一次。形式:$\sum\limits_{i = 0}^{n} p_i(x) y^{(i)}(x) = f(x)$。

        初值条件:点值 $(x_0, y(x_0),\, y^{(1)}(x_0),\, \cdots,\, y^{(n - 1)}(x_0))$。($n$ 阶导数是变量,故任意指定仍可有解。)\\
        {\heiti 解的存在唯一性定理}:若 $\{p_n(x)\}, f(x)$ 均在区间 $I$ 上连续,则存在{\heiti 唯一}满足初值条件的解,
        其中 $x_0 \in I$。

        线性微分算子:$L() = \sum\limits_{i = 0}^{n} p_i(x) \dfrac{\dif^i}{\dif x^i} ()$。不难发现满足:\\
        a. $L(0) = 0$\\
        b. $L(u + v) = L(u) + L(v)$\\
        c. $L(\lambda u) = \lambda L(u), \; \lambda \in \field{R}$\\
        为 $\poly \to \poly$ 的线性映射。

        线性相关:$\exists \lambda_{1 \sim n} \in \field{R}$,使得 $\sum \lambda_i f_i(x) \equiv 0$ 在区间 $I$ 上恒成立。
        线性无关同理。

        判定线性相关性:是否存在一列不全为零的实数 $\{\lambda_n\}$,使得 $\sum\lambda_i f_i(x) \equiv 0$?\\
        将恒等式两边分别求导,仍为恒等式,则 $\forall k$,$\sum\lambda_i f_i^{(k)}(x) \equiv 0$。
        让 $k$ 取遍 $0 \sim n - 1$,则得 $n$ 个等式。\\
        考虑带入一点 $x = x_0$,则所有函数化为常数,变为线性方程组 $\mat{A\lambda} = \mat{0}$,由此引出行列式。

        Wronski 行列式:
        $$w(x_0) = 
        \begin{vmatrix}
            f_1(x_0) & f_2(x_0) & \cdots & f_n(x_0) \\
            f_1^{(1)}(x_0) & f_2^{(1)}(x_0) & \cdots & f_n^{(1)}(x_0) \\
            \vdots & \vdots & \ddots & \vdots \\
            f_1^{(n - 1)}(x_0) & f_2^{(n - 1)}(x_0) & \cdots & f_n^{(n - 1)}(x_0)
        \end{vmatrix}$$
        $\{f_n\}$ 在 $I$ 上线性无关则 $w(x_0) \neq 0$。
        进一步地,若线性无关则 $\forall x_0 \in I$,$w_(x_0) \neq 0$。故 $w(x_0)$ 要么恒非零,要么恒为零。\\
        若{\heiti 限制}函数均为{\heiti 线性微分方程的解},则 $w(x_0) \neq 0$ 说明 $\{f_n\}$ 线性无关。即该条件{\heiti 充要}。

        证明:充分性:线性无关,则任意点处向量 $(f_i(x_0),\, f_i^{(1)}(x_0), \, \cdots \, f_i^{(n - 1)}(x_0))$ 也都线性无关。
        因此方程组有且仅有零解,即 $\det \neq 0$。\\
        必要性:说明有非零解既线性相关。在该点解出一组 $\{\lambda_n\}$,但需说明在所有点处均成立。记 $f^\ast = \sum\lambda_i f_i(x)$,
        为原线性微分方程满足初值条件 $(x_0, 0, \cdots, 0)$ 的解。
        而 $y = 0$ 亦为解,由解的存在唯一性定理知 $f^\ast \equiv 0$,因而线性相关。

        解的结构:零空间的平移,零解 + 特解。

        零空间:$\dim \ker L = n$。\\
        证明:任取 $x_0 \in I$,取 $n$ 个初值条件 $\{\mat{v}_n\}$,其中 $\mat{v}_i$ 除指定 $x_0, y^{(i)} = 1$ 外均为零。
        知存在 $n$ 个解 $\{y_n\}$,而显然 $w(x_0) = 1$,故线性无关。又 $\dim$ 不会大于 $n$,于是恰为 $n$。

        进一步地,$L(u) = f(x), \; L(v) = g(x) \; \implies \; L(u + v) = f(x) + g(x)$。\\
        因此本质上只需对每个单项式求特解,加起来即可。

        \subsection{常系数线性方程}
        常系数:$\forall p_i(x) = a_i \in \field{R}$。形式 $\sum\limits_{i = 0}^{n} a_i y^{(i)} = f(x)$

        先求零空间。猜测有形如 $y = \exp(\lambda x)$ 的解。\\
        代入:$\sum\limits_{i = 0}^{n} a_i \lambda^i \exp(\lambda x) = 0 \;\implies\; \sum a_i \lambda^i = 0$。

        特征方程:$\sum a_i \lambda^i = 0$,以幂换导。\\
        特征根/特征值:$\lambda$,代数基本定理保证 $\field{C}$ 内恰有 $n$ 个根。

        定义 $\exp(z) := \lim_{n \to \infty} (1 + \frac{z}{n})^n$。可证明极限对任意 $z$ 均存在。\\
        欧拉公式:$$\exp(\imunit \theta) = \cos\theta + \imunit\sin\theta$$
        对于一对共轭复根 $z = \alpha \pm \beta\imunit$,取其等价的线性组合代替之,使全部运算限制在 $\field{R}$ 内:
        $$\begin{aligned}
            \frac{1}{2}(\exp(zx) + \exp(\bar{z}x)) = \exp\cos(\beta x) \\
            \frac{1}{2\imunit}(\exp(zx) + \exp(\bar{z}x)) = \exp\sin(\beta x)
        \end{aligned}$$

        对于 $k$ 重根 $\lambda$,仅知一解 $\exp(\lambda x)$。为求其他线性无关解,可设 $y(x) = u(x)\exp(\lambda x)$,
        其中 $u(x)$ 为一非常值函数(否则必然被 $\exp(\lambda x)$ 包含)。不妨指定 $a_n = 1$ 使特征多项式首一。将 $y(x)$ 带入方程:
        $$\begin{aligned}
            \sum_{i = 0}^{n} a_i (u(x)\exp(\lambda x))^{(i)}
            & = \sum_{i = 0}^{n} a_i \sum_{j = 0}^{i} \binom{i}{j} u^{(j)} \lambda^{i - j} \exp(\lambda x) & \text{(Leibniz)} \\
            & = \exp(\lambda x) \sum_{j = 0}^{n} u^{(j)} \lambda^{-j} \sum_{i = j}^{n} a_i \binom{i}{j} \lambda^{i}
        \end{aligned}$$

        $|\exp(\alpha + \beta\imunit)| = |\exp(\alpha)(\cos\beta + \imunit\sin\beta)| = \exp(\alpha)$,
        故即使在 $\field{C}$ 内 $\exp(x)$ 仍恒非零。\\
        手玩若干简单例子发现,$\forall t = 0 \sim k - 1$,$u^{(t)}$ 的系数总为零,但对更高次导数的系数则未必。
        如果这条结论成立,则任取 $u$ 满足 $u^{(k)} = 0$ 均能使方程成立。
        取 $u = x, \, x^2, \, \cdots, \, x^{k - 1}$。它们及 $u = 1$ 时的 $k$ 个解显然线性无关。

        试证明之。$\sum a_i \lambda^i = \prod (\lambda - \lambda_i)$,于是 $a_i$ 即为所有选 $n - i$ 个 $-\lambda_j$ 的乘积之和。
        记 $b_t$ 为 $u^{(t)}\lambda^{-t}$ 的系数。对 $0 \leqslant t < k$,有:
        $$b_t = \sum_{i = t}^{n} \binom{i}{t} \lambda^{i} \sum_{|S| = n - i} \prod_{j \in S} (-\lambda_j)$$

        证明为零,自然地考虑抵消。在 $S$ 中选择 $\lambda$ 自带负号,而 $\lambda^i$ 中没有。不妨将 $\lambda$ 数量相同的项一起考虑。
        记 $T = \{j \in S: \, \lambda_j \neq \lambda\}$。若能抵消,则两项的 $T$ 很可能相同,毕竟其他根没有受到任何限制。
        另一方面,$|T| \leqslant n - k$,而最大的 $|S| = n - t > n - k$,说明任意 $T$ 总能出现在某个 $S$ 中。
        再记 $U = \{j \in [1, n] \cap \field{N}: \, \lambda_j \neq \lambda\}$。换言之,如果在最外层求和中枚举 $T$,
        则 $T$ 自然遍历 $U$ 的所有子集。按直觉这样是对的。于是有:
        $$\begin{aligned}
            b_t & = \sum_{i = t}^{n} \binom{i}{t} \lambda^i \sum_{p = 0}^{\min\{n - p, k\}} \binom{k}{p}(-\lambda)^{p}
            \sum_{T \subset U, |T| = n - i - p} \prod_{j \in T} (-\lambda_j) & \\
            & = \sum_{i = t}^{n} \binom{i}{t} \lambda^i \sum_{p = 0}^{n - p} \binom{k}{p}(-\lambda)^{p}
            \sum_{T} \prod (-\lambda_j) & \text{(}p > k\text{ 时自然为零)} \\
            & = \sum_{T \subset U} \prod_{j \in T} (-\lambda_j) 
            \sum_{p = 0}^{\min\{n - t, n - |T| - t\}} \binom{k}{p} (-\lambda)^p \binom{n - |T| - p}{t} \lambda^{n - |T| - p}
            & \text{(}i = n - |T| - p \geqslant t\text{)} \\
            & = \sum_{T \subset U} \prod_{j \in T} (-\lambda_j) \lambda^{n - |T|} 
            \sum_{p = 0}^{n - |T| - t} (-1)^p \binom{k}{p} \binom{n - |T| - p}{t}
        \end{aligned}$$

        记 $m = n - |T| \geqslant k$,
        $w = \sum\limits_{p = 0}^{m - t} (-1)^p \binom{k}{p}\binom{m - p}{t} 
        = \sum\limits_{p \leqslant m} (-1)^p \binom{k}{p}\binom{m - p}{t}$(放宽限制)。似乎唯一的希望在于 $w \equiv 0$。
        程序验证大有裨益:对充分多的 $t < k \leqslant m$,上式均为零。
        特别地,任一不等号不成立则 $w \not\equiv 0$,以及 $t = k \leqslant m$ 时 $w \equiv 1$。

        世外高人 {\color{red}{$\text{M}$}}{\color{black}{$\text{atrixGroup}$}} 指出
        $$w = [x^m](1 - x)^k\frac{x^t}{(1 - x)^{t + 1}}$$

        解读:枚举前一项对幂次的贡献 $p$,对应于 $(-1)^p\binom{k}{p}$。
        改写第二个组合数,$\binom{m - p}{t} = \binom{m - p}{m - p - t}$。这里 $m - p$ 整体作为变量,是它贡献的幂次。
        上下指标的差恒为 $t$,启示:
        $$\frac{1}{(1 - x)^{t + 1}} = \sum \binom{t + n}{n} x^n$$
        对比原式,应让系数整体向高位偏移 $t$,综上得到 $x^t / (1 - x)^{t + 1}$。

        最后一步已经呼之欲出:
        $$w = [x^m](1 - x)^k\frac{x^t}{(1 - x)^{t + 1}} = [x^{m - t}](1 - x)^{k - t - 1}$$
        当 $t < k \leqslant m$ 时,$k - t - 1 \geqslant 0$ 为多项式,而 $m - t > k - 1 - t$ 严格大于多项式次数,系数显然为零。\\
        ($t = k$ 时,$1 / (1 - x)$ 对应的数列为 $(1, 1, 1, \cdots)$,所以 $w \equiv 1$。)

        \begin{itemize}
            \item $k$ 重实根 $\lambda$:\\
                $\exp(\lambda x), \, x\exp(\lambda x), \, \cdots, \, x^{k - 1}\exp(\lambda x)$
            \item $k$ 重共轭复根 $\alpha \pm \beta\imunit$:
                \begin{longtable}[h]{*{2}{w{c}{4cm}} w{c}{0.5cm} w{c}{4cm}}
                    $\exp(\alpha x)\cos(\beta x)$ & $\exp(\alpha x)\cdot x\cos(\beta x)$ 
                    & $\cdots$ & $\exp(\alpha x)\cdot x^{k - 1}\cos(\beta x)$ \\
                    $\exp(\alpha x)\sin(\beta x)$ & $\exp(\alpha x)\cdot x\sin(\beta x)$ 
                    & $\cdots$ & $\exp(\alpha x)\cdot x^{k - 1}\sin(\beta x)$
                \end{longtable}
        \end{itemize}

        特解:待定系数。$k$ 重根补 $x^k$ 的证明同上,低 $k$ 位系数均零。
        \begin{itemize}
            \item $F(x) = \varphi(x)\exp(\mu x)$,其中 $\varphi(x) \in \mathrm{Poly}$:\\
                $y^\ast = x^k P(x) \exp(\mu x)$,\\
                其中 $\mu$ 为 $k$ 重实根,$P(x) \in \mathrm{Poly}, \deg P(x) = \deg\varphi(x)$。
            \item $F(x) = \varphi(x)\exp(\mu x)\cos(\nu x) \, / \, \varphi(x)\exp(\mu x)\sin(\nu x)$,
                其中 $\varphi(x) \in \mathrm{Poly}$:\\
                $y^\ast = x^k\exp(\mu x)[P_1(x)\cos(\nu x) + P_2(x)\sin(\nu x)]$,\\
                其中 $\mu \pm \nu\imunit$ 为 $k$ 重共轭复根,
                $P_1(x),P_2(x) \in \mathrm{Poly}, \deg P_1(x) = \deg P_2(x) = \deg\varphi(x)$。
        \end{itemize}

        \subsection{Euler 方程}
        形式:
        $$\sum_{i = 0}^{n} a_i x^i \frac{\dif^i}{\dif x^{i}} y^{(i)} = f(x)$$

        换元 $x = \exp(t)$,即 $t = \ln(x)$。可以消去幂次。

    \section{线性微分方程组}
    没学,略。

    $\underbrace
    {
        \widehat
        {
            \overline
            {
                \left|
                \stackrel {\ } {\ \cup}
                \stackrel {\bigcap ^ \cup} {}
                \stackrel {\ } {\cup ^ {\ }}
                \stackrel {\bigcap_\cup} {}
                \right|
            }
        }
    }$

\end{document}
